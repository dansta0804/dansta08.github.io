\documentclass[12pt]{article}
\usepackage{indentfirst}
\usepackage[utf8x]{inputenc}
\usepackage[T1]{fontenc}
\usepackage[english,lithuanian]{babel}
\usepackage{array}
\usepackage{caption}
\usepackage{subcaption}
\usepackage{makecell}
\usepackage[euler]{textgreek}
\usepackage{multirow}
\usepackage{boldline}
\usepackage{floatrow}
\floatsetup[table]{capposition=top}
\usepackage{amsmath, amsthm, amssymb}
\usepackage{graphicx}
\usepackage{setspace}
\usepackage{verbatim}
\usepackage[left=3cm,top=2cm,right=1.5cm,bottom=2cm]{geometry}
\usepackage{floatrow}
\newfloatcommand{capbtabbox}{table}[][\FBwidth]
\usepackage{blindtext}
\onehalfspacing
\usepackage[hidelinks, unicode]{hyperref}
\usepackage{textcomp}
\usepackage{amsmath}
\usepackage{cleveref}
\usepackage[labelfont=bf]{caption}
\usepackage{microtype}
\usepackage{tabularx}
\captionsetup[table]{font={normalfont},format=plain,labelsep=period}
\graphicspath{{../assets/images/}}
\usepackage[table]{xcolor}
\definecolor{newGray}{rgb}{0.878, 0.878, 0.878}
\usepackage{longtable}
\usepackage{enumitem}
\definecolor{deepchampagne}{rgb}{0.98, 0.84, 0.65}
\definecolor{dartmouthgreen}{rgb}{0.09, 0.45, 0.27}
\definecolor{deepcarmine}{rgb}{0.66, 0.13, 0.24}
\definecolor{steelblue}{rgb}{0.27, 0.51, 0.71}
\usepackage{makecell}

\newcommand{\EE}{\mathbb{E}\,}
\newcommand{\ee}{{\mathrm e}}
\newcommand{\dd}{{\mathrm d}}
\newcommand{\RR}{\mathbb{R}}

\begin{document}
\selectlanguage{lithuanian}

\begin{titlepage}
\vskip 20pt
\begin{center}
\includegraphics[scale=1.4]{KTU.png}
\end{center}

%%%%%%%%%%%%%%%%%%%%%%%
% TITULINIS PUSLAPIS
%%%%%%%%%%%%%%%%%%%%%%%

\vskip 20pt
\centerline{\bf \large \textbf{Kauno technologijos universitetas}}
\bigskip
\centerline{\large {Informatikos fakultetas}}
\bigskip

\vskip 90pt
\begin{center}
    {\bf \LARGE Modulis „Tiriamasis projektas 3“}
    \vskip 10pt
    {\bf \Large Projektas: „Pseudonimizuota genetinių duomenų keitimosi sistema“}
    \vskip 15pt
    {\large Testavimo planas}
\end{center}

\vskip 40pt

\hskip 200pt {\bf \large IFM 4/2 gr. Danielė Stasiūnaitė}
\vskip 1pt
\hskip 200pt {\large Studentė}
\vskip 7pt
\hskip 200pt {\bf \large Doc. Mindaugas Vasiljevas}
\vskip 1pt
\hskip 200pt {\large Projekto vadovas}
\vskip 7pt
\hskip 200pt {\bf \large Doc. dr. Eglė Butkevičiūtė}
\vskip 1pt
\hskip 200pt {\large Dėstytoja}

\bigskip

\vskip 100pt
\centerline{\large \textbf{Kaunas, 2025}}
\newpage
\end{titlepage}

\selectlanguage{lithuanian}

%%%%%%%%%%%%%%%%%%%%%
% TURINIO PUSLAPIS
%%%%%%%%%%%%%%%%%%%%%

\tableofcontents
\newpage

\section{Įvadas}
Šiame dokumente yra aprašytas testavimo planas, kuris yra skirtas patikrinti, ar
pseudonimizuota genetinių duomenų keitimosi sistema, skirta darbui su asmenų
biologiniais duomenimis, veikia taip, kaip yra numatyta parengtoje sistemos
specifikacijoje.

\subsection{Testavimo tikslai ir objektai}
Pagrindinis šio dokumento tikslas yra apibrėžti testavimo strategiją,
apimtį, metodus ir išteklius, kurie yra reikalingi efektyviam pseudonimizuotos
genetinių duomenų keitimosi sistemos testavimui. Pasinaudojus parengtu testavimo
planu siekiama įvertinti, ar sukurta sistema tenkina sistemai iškeltus
funkcinius ir nefunkcinius sistemos reikalavimus. Taip pat su šiuo dokumentu
siekiama numatyti aiškų planą, kokie sistemos komponentai bus testuojami,
kaip bus identifikuojamos sistemos klaidos ir kur jos bus fiksuojamos, kad su
vėlesnėmis sistemos versijomis galėtų būti ištaisytos.

\subsection{Testavimo apimtis ir tipai}
\label{sec:SCOPE}
Numatyta, kad sistemos testavimas bus vykdomas, atliekant visų sistemos
funkcinių dalių testavimą, kur funkcinės dalys apima:
\begin{itemize}
    \item Registracijos modulio veikimą (paskyrų kūrimą skirtingoms sistemos
    naudotojų grupėms: pacientams, gydytojams - genetikams, tyrėjams);
    \item Sistemos naudotojų - pacientų - biologinių duomenų prieigos valdymą;
    \item Duomenų šifravimo, pseudonimizavimo ir dešifravimo funkcijas;
    \item Biologinių duomenų analizės funkcionalumą;
    \item Užklausų biologinės analizės atlikimui generavimą;
    \item Analizės rezultatų pateikimą.
\end{itemize}

Aprašytos pagrindinės sistemos funkcinės dalys bus testuojamos, taikant šias
testavimo strategijas:
\begin{itemize}
    \item \textbf{Vienetų testavimą (angl. \emph{unit testing})}, kai bus
    testuojami atskiri sistemos komponentai. T.y., bus tikrinama, ar kiekvienas
    sukurtas sistemos modulis (pacientų, gydytojų - genetikų, tyrėjų) ir
    funkcinė dalis veikia pagal parengtą sistemos specifikaciją.
    \item \textbf{Integracijos testavimą (angl. \emph{integration testing})},
    kai bus testuojama, kaip atskiri sistemos moduliai sąveikauja tarpusavyje,
    ar duomenys yra korektiškai perduodami tarp skirtingų sistemos modulių.
    \item \textbf{Sistemos testavimą (angl. \emph{system testing})}, kai bus
    testuojama, ar sistema veikia kaip vieninga visuma, ar sistema atitinka
    iškeltus nefunkcinius reikalavimus.
    \item \textbf{Priėmimo testavimą (angl. \emph{acceptance testing})}, kai bus
    tikrinama, ar sistema atitinka visus užsakovo reikalavimus. Į šį testavimą
    bus įtrauktas ir sistemos užsakovas.
    \item \textbf{Saugumo testavimą}, kai bus tikrinama, ar sistema yra
    pasiekiama tik autorizuotiems sistemos naudotojams ir ar sistemoje
    apdorojami jautrūs biologiniai duomenys negali būti tiesiogiai prieinami
    naudotojams (ar duomenų negalima atsisiųsti, ar negalima peržiūrėti
    biologinių duomenų failų informacijos be autorizuoto paciento leidimo).
\end{itemize}

\subsection{Pagrindiniai apribojimai}
Žemiau aprašyti testavimo aplinkos - aparatinės ir programinės įrangos bei
testavimo įrankių - konfigūracijos apribojimai, kurie turi būti įgyvendinti,
siekiant užtikrinti efektyvų bei kokybišką testavimą.

\vskip 10pt

\noindent \textbf{Aparatinė įranga}
\begin{itemize}
    \item Turi būti naudojamas serveris, turintis 8-16 branduolių CPU,
    32-64 GB operatyvios atminties (RAM) ir SSD diską su minimalia 512 GB
    atmintimi.
    \item Testavimui turi būti naudojami kompiuteriai, turintys Windows 10/11
    arba Linux (Ubuntu 20.04+) operacines sistemas.
\end{itemize}

\noindent \textbf{Programinė įranga}
\begin{itemize}
    \item Duomenų saugojimui turi būti naudojama MySQL 8.0 arba vėlesnės
    versijos duomenų bazės valdymo sistema.
    \item \emph{Back-end} API kūrimui ir testų rašymui turi būti naudojama
    Python 3.10 arba vėlesnė versija.
\end{itemize}

\noindent \textbf{Testavimo įrankiai}
\begin{itemize}
    \item Vienetų ir integracijos testavimui atlikti turi būti naudojama Python
    testavimo biblioteka PyTest.
    \item Naudotojo sąsajos (angl. \emph{User Interface} (UI)), sukurtos su R
    programavimo kalbos Shiny biblioteka, testavimui turi būti naudojamas
    automatizuotas naršyklės valdymo įrankis Selenium.
    \item API testavimui turi būti naudojamas Postman įrankis.
    \item Sistemos saugumo testavimui atlikti turi būti naudojamas SoapUI
    įrankis.
    \item Sistemos našumo testavimui atlikti turi būti naudojamas Apache JMeter
    įrankis.
\end{itemize}

\newpage

\subsection{Nuorodos}
Žemiau pateiktoje lentelėje aprašyti visi su šiame dokumente aprašoma testuojama
sistema susiję dokumentai.

\label{sec:NUORODOS}
\begin{table}[htb!]
    \captionsetup{justification=centering}
    \caption{\small\textbf{Susiję dokumentai}.}
    \vskip -10pt
    \begin{tabular}{
        |>{\centering\arraybackslash}m{0.7cm}
        |>{\centering\arraybackslash}m{13cm}|
    }
        \hline
        \textbf{\cellcolor{deepchampagne}Eil. Nr.} &
        \textbf{\cellcolor{deepchampagne}Dokumento pavadinimas ir nuoroda} \\
        \hline
        \multicolumn{1}{|>{\arraybackslash}m{0.7cm}|}{1.} &
        \multicolumn{1}{>{\raggedright\arraybackslash}m{13cm}|}
        {\color{steelblue}
        \href{https://dansta0804.github.io/dansta08.github.io/PDFs/Literat\%C5\%ABros\_analiz\%C4\%97.pdf}
        {Projektavimo metodologijos ir technologijų analizė}} \\
        \hline
        \multicolumn{1}{|>{\arraybackslash}m{0.7cm}|}{2.} &
        \multicolumn{1}{>{\raggedright\arraybackslash}m{13cm}|}
        {\color{steelblue}
        \href{https://dansta0804.github.io/dansta08.github.io/PDFs/Projekto\_parai\%C5\%A1ka.pdf}
        {Projekto paraiška}} \\
        \hline
        \multicolumn{1}{|>{\arraybackslash}m{0.7cm}|}{3.} &
        \multicolumn{1}{>{\raggedright\arraybackslash}m{13cm}|}
        {\color{steelblue}
        \href{https://dansta0804.github.io/dansta08.github.io/PDFs/Projekto\_planas\_V2.pdf}
        {Projektavimo planas}} \\
        \hline
        \multicolumn{1}{|>{\arraybackslash}m{0.7cm}|}{4.} &
        \multicolumn{1}{>{\raggedright\arraybackslash}m{13cm}|}
        {\color{steelblue}
        \href{https://dansta0804.github.io/dansta08.github.io/PDFs/Reikalavim\%C5\%B3\_specifikavimas.pdf}
        {Sistemos reikalavimų specifikacija}} \\
        \hline
        \multicolumn{1}{|>{\arraybackslash}m{0.7cm}|}{5.} &
        \multicolumn{1}{>{\raggedright\arraybackslash}m{13cm}|}
        {\color{steelblue}\emph{\href{https://dansta0804.github.io/dansta08.github.io/PDFs/Projekto\_architekt\%C5\%ABra.pdf}
        {Sistemos architektūros specifikacija}}} \\
        \hline
    \end{tabular}
\end{table}

\newpage

\section{Testavimo procedūra}
\subsection{Pradinės sąlygos}
Tam, jog galėtų būti pradėtas testavimo etapas, turi būti tenkinamos pradinės
testavimo sąlygos, kur turi būti:

\begin{itemize}
    \item Pilnai paruošta ir patvirtinta sistemos reikalavimų specifikacija,
    pagal kurią būtų galima testuoti sistemos funkcionalumą.
    \item Paruošti ir patvirtinti testavimo scenarijai, pagal kuriuos bus
    atliekamas testavimas.
    \item Paruošti testavimo duomenys, kurie bus naudojami testavimo metu
    (skirtingų kategorijų sistemos naudotojų paskyros, skirtingi biologinių
    duomenų failai, skirtingi analizės atlikimo užklausų pavyzdžiai).
    \item Pilnai paruošta ir sukonfigūruota testavimo aplinka, kurioje bus
    atliekamas sistemos testavimas (duomenų bazės paruošimas, reikalingų
    programinės įrangos komponentų ir testavimo įrankių įdiegimas ir
    konfigūravimas).
    \item Paruoštas testavimo planas, kuriame pateikta visa su testavimu
    susijusi informacija bei pateikti testavimo scenarijai, kuriuose turės būti
    fiksuojamas testavimo rezultatas.
    \item Paruoštas testavimo klaidų ataskaitos šablonas, pagal kurį bus
    fiksuojamos testavimo metu identifikuotos klaidos.
\end{itemize}

\subsection{Testavimo prioritetai}
Testavimas turi būti atliekamas pagal šiuos nustatytus prioritetus:

\begin{itemize}
    \item \textbf{Iš pradžių turi būti testuojami sistemos naudotojui matomi
    sistemos funkciniai reikalavimai}:
    \begin{itemize}[label=$\circ$]
        \item Ar naujas sistemos naudotojas gali užsiregistruoti sistemoje kaip
        pacientas, gydytojas - genetikas arba tyrėjas.
        \item Ar prisijungęs naudotojas pagal savo kategoriją gali pasiekti jam
        skirtą funkcionalumą. Pavyzdžiui, ar pacientas gali įkelti savo
        biologinius duomenis ir valdyti, kokie asmenys gali šiuos duomenis
        pasiekti.
        \item Ar gydytojas - genetikas gali sukurti pacientų medicininės
        kortelės įrašus, pasiekti pacientų biologinius duomenis ir sugeneruoti
        analizės atlikimo užklausą tyrėjui.
        \item Ar tyrėjas gali pasiekti atitinkamų pacientų biologinius duomenis
        (jei iš gydytojo -  genetiko yra gavęs analizės atlikimo užklausą), ar
        gali atlikti biologinių duomenų analizę ir perduoti analizės atlikimo
        rezultatus gydytojui - genetikui.
    \end{itemize}
    \item \textbf{Toliau turi būti testuojami sistemos naudotojui nematomi
    sistemos funkciniai reikalavimai}:
    \begin{itemize}[label=$\circ$]
        \item Ar korektiškai veikia naudotojo autentifikavimo funkcijos.
        \item Ar sistemos naudotojų - pacientų - biologiniai duomenys yra
        korektiškai šifruojami, pseudonimizuojami ir dešifruojami.
    \end{itemize}
    \item \textbf{Po funkcinių reikalavimų testavimo vykdomas aukšto prioriteto
    sistemos ne\-funk\-ci\-nių reikalavimų testavimas}:
    \begin{itemize}[label=$\circ$]
        \item Ar autentifikacijos operacijos atlikimo greitis neviršija 1
        sekundės.
        \item Ar paciento biologinių duomenų prieigos patikrinimas netrunka
        ilgiau nei 500 milisekundžių.
    \end{itemize}
    \item \textbf{Galiausiai vykdomas žemo prioriteto sistemos nefunkcinių
    reikalavimų testavimas}:
    \begin{itemize}[label=$\circ$]
        \item Ar naudotojo sąsajos elementai yra tinkamai išdėstyti.
        \item Ar sistemoje realizuoti iššokantys langeliai netrukdo ir neerzina.
        \item Ar sistema yra intuityvi ir pritaikyta vyresnio amžiaus žmonėms.
        \item Ar sistemoje yra pateikti skirtingų techninių terminų bei
        privalomų įvesties laukų paaiškinimai.
    \end{itemize}
\end{itemize}

\subsection{Testavimo rezultatų kaupimas}
Atlikus testavimą bus pilnai užpildyti arba sukurti dokumentai - testavimo
rezultatai:

\begin{itemize}
    \item Bus pilnai užpildytas testavimo planas (prie testavimo scenarijų
    nurodant testavimo rezultatą ir testo būseną);
    \item Bus sukurta testavimo klaidų ataskaita;
    \item Su užsakovu bus pasirašytas dokumentas, patvirtinantis, kad sistema
    atitinka visus užsakovo sistemai iškeltus reikalavimus.
\end{itemize}

\newpage

\subsection{Testavimo tvarkaraštis}
Žemiau pateiktoje lentelėje yra pateiktas testavimo tvarkaraštis, kuriame
nurodyta skirtingų testavimo etapų pradžia ir pabaiga.

\begin{table}[htb!]
    \captionsetup{justification=centering}
    \caption{\small\textbf{Testavimo tvarkaraštis.}}
    \vskip -10pt
    \begin{tabular}{
        |>{\centering\arraybackslash}m{5cm}
        |>{\centering\arraybackslash}m{4cm}
        |>{\centering\arraybackslash}m{4cm}|
    }
        \hline
        \textbf{\cellcolor{deepchampagne}Testavimo metodas} &
        \textbf{\cellcolor{deepchampagne}Pradžia} &
        \textbf{\cellcolor{deepchampagne}Pabaiga}  \\
        \hline
        \multicolumn{1}{|>{\raggedright\arraybackslash}m{5cm}|}
            {Vienetų testavimas} &
        \multicolumn{1}{>{\raggedright\arraybackslash}m{4cm}|}{2025-11-03} &
        \multicolumn{1}{>{\raggedright\arraybackslash}m{4cm}|}{2025-11-10}\\
        \hline
        \multicolumn{1}{|>{\raggedright\arraybackslash}m{5cm}|}
            {Integracijos testavimas} &
        \multicolumn{1}{>{\raggedright\arraybackslash}m{4cm}|}{2025-11-10} &
        \multicolumn{1}{>{\raggedright\arraybackslash}m{4cm}|}{2025-11-17}\\
        \hline
        \multicolumn{1}{|>{\raggedright\arraybackslash}m{5cm}|}
            {Saugumo testavimas} &
        \multicolumn{1}{>{\raggedright\arraybackslash}m{4cm}|}{2025-11-17} &
        \multicolumn{1}{>{\raggedright\arraybackslash}m{4cm}|}{2025-11-24}\\
        \hline
        \multicolumn{1}{|>{\raggedright\arraybackslash}m{5cm}|}
            {Sistemos testavimas} &
        \multicolumn{1}{>{\raggedright\arraybackslash}m{4cm}|}{2025-11-19} &
        \multicolumn{1}{>{\raggedright\arraybackslash}m{4cm}|}{2025-12-15}\\
        \hline
        \multicolumn{1}{|>{\raggedright\arraybackslash}m{5cm}|}
            {Priėmimo testavimas} &
        \multicolumn{1}{>{\raggedright\arraybackslash}m{4cm}|}{2025-12-15} &
        \multicolumn{1}{>{\raggedright\arraybackslash}m{4cm}|}{2025-12-22}\\
        \hline
    \end{tabular}
    \label{table:TESTAVIMO_KALENDORIUS}
\end{table}

\newpage

\section{Testavimo scenarijai}
Žemiau aprašyti keli sistemos funkcionalumo testavimo scenarijai, pagal kuriuos
bus atliekamas sukurtos sistemos testavimas.

\begin{table}[htb!]
    \captionsetup{justification=centering}
    \caption{\small\textbf{Testavimo scenarijus Nr. 1.}}
    \vskip -10pt
    \begin{tabular}{|m{6cm}|m{11cm}|}
        \hline
        \raggedleft \textbf{\cellcolor{deepchampagne}Kodas:} &
        \ttfamily{TS\_001}. \\
        \hline
        \raggedleft \textbf{\cellcolor{deepchampagne}Pavadinimas:} & Naudotojo
        (paciento) paskyros sukūrimo testas. \\
        \hline
        \raggedleft \textbf{\cellcolor{deepchampagne}Tikslas:} &
        Patikrinti, ar sistemos svečias gali susikurti sistemos paskyrą kaip
        sistemos naudotojas - pacientas. \\
        \hline
        \raggedleft \textbf{\cellcolor{deepchampagne}Pradinės testavimo
        sąlygos:} & 
        Sistemos svečias turi būti atsidaręs sistemos paskyros kūrimo langą. \\
        \hline
        \raggedleft \textbf{\cellcolor{deepchampagne}Įvestis:}
        & Sistemos svečias turi įvesti savo vardą, pavardę, el. paštą,
        telefono numerį, lytį, adresą ir sugalvotą slaptažodį. \\
        \hline
        \raggedleft \textbf{\cellcolor{deepchampagne}Testo etapai\footnote{Čia
        ir toliau \textcolor{dartmouthgreen}{žalia} spalva pažymėti naudotojo
        veiksmai.}:} & \vskip 5pt
        \makecell[l]{\parbox[t]{11cm}{
            \textbf{1.} \textcolor{dartmouthgreen}{Užpildomi pateiktos
            asmeninės paskyros kūrimo formos laukai.} \\
            \textbf{2.} \textcolor{dartmouthgreen}{Išsaugoma įvesta informacija,
            paspaudžiant išsaugojimo mygtuką.} \\
            \textbf{3.} {Parodomas informacinis pranešimas, informuojantis apie
            sėkmingai sukurtą asmeninę paskyrą.} \\
            \textbf{4.} {Sistema prijungia naudotoją prie jo asmeninės
            paskyros.} \\
            \textbf{5.} {Sistema atidaro naudotojo asmeninės paskyros langą.}
        }} \\
        \hline
        \raggedleft \textbf{\cellcolor{deepchampagne}Tikėtinas rezultatas:}
        & Sukurtas naujas sistemos naudotojas, kuris priklauso kategorijai
        „Pacientas“. \\
        \hline
        \raggedleft \textbf{\cellcolor{deepchampagne}Tikras rezultatas:}
        & \textcolor{red}{\emph{Užpildoma testavimo metu...}} \\
        \hline
        \raggedleft \textbf{\cellcolor{deepchampagne}Būsena:}
        & \textcolor{red}{\emph{Užpildoma testavimo metu: „Testas sėkmingai
        įvykdytas“ arba „Gauta klaida“}}. \\
        \hline
    \end{tabular}
    \label{table:TS_1}
\end{table}

\newpage

\begin{table}[htb!]
    \captionsetup{justification=centering}
    \caption{\small\textbf{Testavimo scenarijus Nr. 2.}}
    \vskip -10pt
    \begin{tabular}{|m{6cm}|m{11cm}|}
        \hline
        \raggedleft \textbf{\cellcolor{deepchampagne}Kodas:} &
        \ttfamily{TS\_002}. \\
        \hline
        \raggedleft \textbf{\cellcolor{deepchampagne}Pavadinimas:} & Biologinių
        duomenų įkėlimo testas. \\
        \hline
        \raggedleft \textbf{\cellcolor{deepchampagne}Tikslas:} &
        Patikrinti, ar sistemos naudotojas - pacientas - gali įkelti savo
        biologinius duomenis į sistemą. \\
        \hline
        \raggedleft \textbf{\cellcolor{deepchampagne}Pradinės testavimo
        sąlygos:} & 
        Sistemos naudotojas - pacientas - turi būti prisijungęs prie sistemos
        ir atsidaręs biologinių duomenų įkėlimo langą. \\
        \hline
        \raggedleft \textbf{\cellcolor{deepchampagne}Įvestis:}
        & Genetinę informaciją aprašantis failas su kokybiniais įverčiais
        (\emph{.bed, .vcf}). \\
        \hline
        \raggedleft \textbf{\cellcolor{deepchampagne}Testo etapai:} & \vskip 5pt
        \makecell[l]{\parbox[t]{11cm}{
            \textbf{1.} \textcolor{dartmouthgreen}{Užpildomi pateiktos
            duomenų įkėlimo formos laukai ir pridedamas biologinius duomenis
            saugantis failas.} \\
            \textbf{2.} \textcolor{dartmouthgreen}{Išsaugoma įvesta
            metainformacija bei pridėtas failas, paspaudžiant išsaugojimo
            mygtuką.} \\
            \textbf{3.} Sistema validuoja failo formatą ir turinį. \\
            \textbf{4.} Sistema užšifruoja duomenis ir išsaugo juos duomenų
            bazėje. \\
            \textbf{5.} Sistema priskiria įrašui identifikatorių ir susieja jį
            su naudotojo paskyra. \\
            \textbf{6.} Parodomas informacinis pranešimas, informuojantis apie
            sėkmingai įkeltus duomenis. \\
            \textbf{7.} \textcolor{dartmouthgreen}{Peržiūrimas įkeltų
            duomenų įrašas paciento asmeninės paskyros skiltyje.}
        }} \\
        \hline
        \raggedleft \textbf{\cellcolor{deepchampagne}Tikėtinas rezultatas:}
        & Failas sėkmingai įkeltas, užšifruotas ir išsaugotas sistemoje. \\
        \hline
        \raggedleft \textbf{\cellcolor{deepchampagne}Tikras rezultatas:}
        & \textcolor{red}{\emph{Užpildoma testavimo metu...}} \\
        \hline
        \raggedleft \textbf{\cellcolor{deepchampagne}Būsena:}
        & \textcolor{red}{\emph{Užpildoma testavimo metu: „Testas sėkmingai
        įvykdytas“ arba „Gauta klaida“}}. \\
        \hline
    \end{tabular}
    \label{table:TS_2}
\end{table}

\newpage

\begin{table}[htb!]
    \captionsetup{justification=centering}
    \caption{\small\textbf{Testavimo scenarijus Nr. 3.}}
    \vskip -10pt
    \begin{tabular}{|m{6cm}|m{11cm}|}
        \hline
        \raggedleft \textbf{\cellcolor{deepchampagne}Kodas:} &
        \ttfamily{TS\_003}. \\
        \hline
        \raggedleft \textbf{\cellcolor{deepchampagne}Pavadinimas:} & Biologinių
        duomenų prieigos valdymo testas. \\
        \hline
        \raggedleft \textbf{\cellcolor{deepchampagne}Tikslas:} & Patikrinti, ar
        sistemos naudotojas - pacientas - gali valdyti, kas gali pasiekti jo
        įkeltus biologinius duomenis. \\
        \hline
        \raggedleft \textbf{\cellcolor{deepchampagne}Pradinės testavimo
        sąlygos:} & Sistemos naudotojas - pacientas - turi būti prisijungęs prie
        sistemos ir atsidaręs biologinių duomenų prieigos valdymo langą. \\
        \hline
        \raggedleft \textbf{\cellcolor{deepchampagne}Įvestis:} & Duomenų
        prieigos valdymo lange iš teisių sąrašo reikia pasirinkti konkretiems
        asmenims priskiriamas teises. \\
        \hline
        \raggedleft \textbf{\cellcolor{deepchampagne}Testo etapai:} & \vskip 5pt
        \makecell[l]{\parbox[t]{11cm}{
            \textbf{1.} {Sistema pateikia paciento
            įkeltų biologinių duomenų sąrašą.} \\
            \textbf{2.} \textcolor{dartmouthgreen}{Pasirenkamas konkretus
            biologinių duomenų sąrašo įrašas.} \\
            \textbf{3.} {Sistema pateikia naudotojų, turinčių prieigą prie
            konkrečių biologinių duomenų, sąrašą.} \\
            \textbf{4.} \textcolor{dartmouthgreen}{Redaguojamos suteiktos
            prieigos teisės sistemos naudotojams: pratęsiamas prieigos
            laikotarpis arba atšaukiama prieiga.} \\
            \textbf{5.} \textcolor{dartmouthgreen}{Suteikiamos naujos prieigos
            naujiems sistemos naudotojams.} \\
            \textbf{6.} {Sistema atnaujina naudotojams suteiktų prieigų
            sąrašą.} \\
            \textbf{7.} {Parodomas informacinis pranešimas, informuojantis apie
            sėkmingai atliktą teisių atnaujinimą.} \\
            \textbf{8.} {Sistema informuoja atitinkamus sistemos naudotojus apie
            prieigos teisių pasikeitimus.}
        }} \\
        \hline
        \raggedleft \textbf{\cellcolor{deepchampagne}Tikėtinas rezultatas:}
        & Sistema leidžia sistemos naudotojui - pacientui - valdyti prieigą prie
        duomenų. Asmuo, kuriam suteikiama prieiga prie paciento įkeltų
        biologinių duomenų, turi galėti vykdyti šių duomenų peržiūrą. \\
        \hline
        \raggedleft \textbf{\cellcolor{deepchampagne}Tikras rezultatas:}
        & \textcolor{red}{\emph{Užpildoma testavimo metu...}} \\
        \hline
        \raggedleft \textbf{\cellcolor{deepchampagne}Būsena:}
        & \textcolor{red}{\emph{Užpildoma testavimo metu: „Testas sėkmingai
        įvykdytas“ arba „Gauta klaida“}}. \\
        \hline
    \end{tabular}
    \label{table:TS_3}
\end{table}

\newpage

\begin{table}[htb!]
    \captionsetup{justification=centering}
    \caption{\small\textbf{Testavimo scenarijus Nr. 4.}}
    \vskip -10pt
    \begin{tabular}{|m{6cm}|m{11cm}|}
        \hline
        \raggedleft \textbf{\cellcolor{deepchampagne}Kodas:} &
        \ttfamily{TS\_004}. \\
        \hline
        \raggedleft \textbf{\cellcolor{deepchampagne}Pavadinimas:} & Prašymo
        „būti pamirštam“ pateikimo ir įgyvendinimo testas. \\
        \hline
        \raggedleft \textbf{\cellcolor{deepchampagne}Tikslas:} & Patikrinti, ar
        sistemos naudotojas - pacientas - gali sėkmingai pateikti prašymą
        panaikinti visus su juo susijusius įrašus iš saugyklų. \\
        \hline
        \raggedleft \textbf{\cellcolor{deepchampagne}Pradinės testavimo
        sąlygos:} & Sistemos naudotojas - pacientas - turi būti prisijungęs prie
        sistemos ir atsidaręs biologinių duomenų prieigos valdymo langą. \\
        \hline
        \raggedleft \textbf{\cellcolor{deepchampagne}Įvestis:} & Duomenų
        prieigos valdymo lange reikia pasirinkti visų su naudotoju susijusių
        duomenų panaikinimo parinktį. \\
        \hline
        \raggedleft \textbf{\cellcolor{deepchampagne}Testo etapai:} & \vskip 5pt
        \makecell[l]{\parbox[t]{11cm}{
            \textbf{1.} \textcolor{dartmouthgreen}{Asmeninės paskyros redagavimo
            lange pažymima parinktis „Prašymas būti pamirštam“.} \\
            \textbf{2.} {Sistema pateikia pasekmių, susijusių su prašymo būti
            pamirštam išsiuntimu, sąrašą ir nurodo, kad reikalingas naudotojo
            patvirtinimas.} \\
            \textbf{3.} \textcolor{dartmouthgreen}{Patvirtinama, kad 
            susipažinta su pasekmėmis ir patvirtinamas prašymas.} \\
            \textbf{4.} {Sistema patikrina, ar einamuoju metu nėra atliekama
            paciento pateiktų biologinių duomenų analizė.} \\
            \textbf{5.} {Sistema panaikina naudotojo asmeninius duomenis, visų
            naudotojų prieigas prie biologinių duomenų, ištrina visus su
            naudotoju susijusius duomenis iš duomenų bazių ir užfiksuoja
            „pamiršimo“ įvykį.} \\
            \textbf{6.} {Parodomas informacinis pranešimas, informuojantis apie
            sėkmingai įgyvendintą prašymą būti pamirštam.}
        }} \\
        \hline
        \raggedleft \textbf{\cellcolor{deepchampagne}Tikėtinas rezultatas:}
        & Naudotojo duomenys yra pašalinti, biologiniai duomenys yra
        nebeprieinami kitiems sistemos naudotojams. \\
        \hline
        \raggedleft \textbf{\cellcolor{deepchampagne}Tikras rezultatas:}
        & \textcolor{red}{\emph{Užpildoma testavimo metu...}} \\
        \hline
        \raggedleft \textbf{\cellcolor{deepchampagne}Būsena:}
        & \textcolor{red}{\emph{Užpildoma testavimo metu: „Testas sėkmingai
        įvykdytas“ arba „Gauta klaida“}}. \\
        \hline
    \end{tabular}
    \label{table:TS_4}
\end{table}

\newpage

\begin{table}[htb!]
    \captionsetup{justification=centering}
    \caption{\small\textbf{Testavimo scenarijus Nr. 5.}}
    \vskip -10pt
    \begin{tabular}{|m{6cm}|m{11cm}|}
        \hline
        \raggedleft \textbf{\cellcolor{deepchampagne}Kodas:} &
        \ttfamily{TS\_005}. \\
        \hline
        \raggedleft \textbf{\cellcolor{deepchampagne}Pavadinimas:} & Paciento
        analizės kortelės įrašo sukūrimo testas. \\
        \hline
        \raggedleft \textbf{\cellcolor{deepchampagne}Tikslas:} & Patikrinti, ar
        sistemos naudotojas, gydytojas - genetikas, gali sėkmingai sukurti
        naują paciento kortelės įrašą su informacija apie reikalingą analizę. \\
        \hline
        \raggedleft \textbf{\cellcolor{deepchampagne}Pradinės testavimo
        sąlygos:} & Sistemos naudotojas, gydytojas - genetikas, turi būti
        prisijungęs prie sistemos ir atsidaręs paciento analizės kortelės įrašų
        kūrimo langą. \\
        \hline
        \raggedleft \textbf{\cellcolor{deepchampagne}Įvestis:} & Paciento
        analizės kortelės įrašų kūrimo lange reikia nurodyti, kokiam tyrėjui
        priskirta analizė, koks analizės tipas, analizės įrašo sukūrimo data ir
        laikas. \\
        \hline
        \raggedleft \textbf{\cellcolor{deepchampagne}Testo etapai:} & \vskip 5pt
        \makecell[l]{\parbox[t]{11cm}{
            \textbf{1.} \textcolor{dartmouthgreen}{Užpildomi analizės kortelės
            įrašo formos laukai.} \\
            \textbf{2.} \textcolor{dartmouthgreen}{Išsaugomi įvesti duomenys,
            paspaudžiant išsaugojimo mygtuką.} \\
            \textbf{3.} Parodomas informacinis pranešimas, informuojantis apie
            sėkmingai sukurtą paciento analizės kortelės įrašą.
        }} \\
        \hline
        \raggedleft \textbf{\cellcolor{deepchampagne}Tikėtinas rezultatas:}
        & Gydytojas - genetikas sėkmingai sukūrė paciento analizės kortelės
        įrašą, su kuriuo gali būti atliekami tolimesni veiksmai. \\
        \hline
        \raggedleft \textbf{\cellcolor{deepchampagne}Tikras rezultatas:}
        & \textcolor{red}{\emph{Užpildoma testavimo metu...}} \\
        \hline
        \raggedleft \textbf{\cellcolor{deepchampagne}Būsena:}
        & \textcolor{red}{\emph{Užpildoma testavimo metu: „Testas sėkmingai
        įvykdytas“ arba „Gauta klaida“}}. \\
        \hline
    \end{tabular}
    \label{table:TS_5}
\end{table}

\newpage

% KABUTĖS: „ “
\end{document}